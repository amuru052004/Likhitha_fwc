\documentclass[a4paper,12pt]{article}
\usepackage{graphicx}
\usepackage{amsmath}
\usepackage{geometry}
\usepackage{amsmath, amssymb}
\usepackage{float}
\usepackage{caption}
\usepackage{subcaption}
\usepackage{xcolor}
\usepackage{fancyhdr}
\usepackage{datetime2}
\usepackage{pgfplots}

\definecolor{darkskyblue}{rgb}{0.0, 0.5, 1.0}
\definecolor{skyblue}{RGB}{135, 206, 235}
\usepackage{wrapfig}
\usepackage{circuitikz}

\geometry{a4paper, top=0.7in, left=1in, right=1in, bottom=1in}

\begin{document}

\pagestyle{empty} % Start with empty page style

\thispagestyle{fancy} % Apply fancy style only to the first page
\fancyhf{} % Clear header and footer
\renewcommand{\headrulewidth}{0pt} % Remove header rule

\fancyhead[L]{% Left header
 \includegraphics[width=8cm, height=1.7cm]{1.png} % Adjust dimensions
}
\fancyhead[R]{% Right header
    Name: A.Likhitha\\
    Batch: COMETFWC019 \\
    Date: 27 May 2025 
}

\vspace{10cm}
\begin{center}
   
    {\LARGE \textbf{\textcolor{darkskyblue}{\\  GATE QUESTION \\ EEE 2009 Q13}}}
\end{center}
\vspace{-1cm} %adjust vertical space

\section*{\textcolor{blue}{\\Question}}
Q13)  The complete set of only those Logic Gates designated as Universal Gates is\\

\hspace{-0.5cm}\textbf{Options:}
\begin{enumerate}
    \item NOT, OR and AND Gates
    \item XNOR, NOR and NAND Gates
    \item NOR and NAND Gates
    \item XOR, NOR and NAND Gates
\end{enumerate}
\vspace{1cm}
\section*{\textcolor{blue}{Answer and Explanation}}

\textbf{Answer:} (3) NOR and N
\vspace{0.5cm}

\textbf{Explanation:}

\begin{itemize}
    \item \textbf{NAND AND NOR GATES} \\
\vspace{0.05em}

    A universal gate is a logic gate that can be used to implement any Boolean function without needing to use any other gate.
    
    NOR Gate and NAND Gate are called universal gates because:

    Any logic circuit can be built entirely using only NOR gates or only NAND gates.

    Other gates like NOT, AND, and OR can be constructed using combinations of only NOR gates or only NAND gates.

 \textbf{For example}

    A NOT gate can be implemented using a NOR gate by tying both its inputs together:

        NOR(A, A) = NOT(A)

    An AND gate can be implemented using NAND gates:

        NAND(A, A) = NOT(A)

        AND(A, B) = NOT(NAND(A, B))

\textbf{Hence, Option (C) is correct.}

\end{document}
