\documentclass[12pt]{article}
\usepackage{graphicx}
\usepackage[margin=1in]{geometry}
\usepackage{caption}
\usepackage{xcolor}
\usepackage{float}
\usepackage{amsmath}
\usepackage{array}
\usepackage{multicol}
\usepackage{hyperref}
\usepackage{url}
\usepackage{listings}
\usepackage{courier}

\lstset{
  basicstyle=\ttfamily\footnotesize,
  backgroundcolor=\color{gray!10},
  frame=single,
  breaklines=true,
  captionpos=b
}

\setlength{\columnsep}{1cm}

\begin{document}

% Title Block
\begin{figure}[H]
    \begin{minipage}{0.45\textwidth}
        \includegraphics[width=\textwidth]{i.png} % Replace with your image
    \end{minipage} \hfill
    \begin{minipage}{0.45\textwidth}
        \textbf{Name: Amuru Likhitha} \\
        \textbf{Batch: COMETFWC019} \\
        \textbf{Date: 10 July 2025}
    \end{minipage}
\end{figure}

\begin{center}
    {\LARGE \textbf{\textcolor{cyan}{GATE 2009, ECE Question Number 60}}}
\end{center}
\vspace{1em}

\begin{multicols}{2}

\noindent\textbf{Abstract} \\[0.5em]
This project demonstrates the implementation of segment logic using only NOT and OR gates as described in GATE Q60. It implements outputs g, e, and d using an Arduino UNO.

\vspace{1em}
\noindent\textbf{1. Components}
\begin{table}[H]
\small
\centering
\begin{tabular}{|p{4.2cm}|c|}
\hline
\textbf{Component} & \textbf{Qty} \\
\hline
Arduino UNO & 1 \\
Push Buttons & 4 \\
LEDs & 3 \\
220$\Omega$ Resistors & 7 \\
Breadboard & 1 \\
Jumper Wires & 10 \\
Laptop with Arduino IDE & 1 \\
\hline
\end{tabular}
\caption*{Table 1: List of components used}
\end{table}

\vspace{1em}
\noindent\textbf{2. Setup and Connections}
\begin{itemize}
    \item Connect buttons P1, P2, b, c to D2, D3, D4, and D5.
    \item Connect LEDs to D8 (g), D9 (e), and D10 (d) via 220$\Omega$ resistors.
    \item Use pull-down resistors for button pins.
    \item Ensure common GND for Arduino and components.
\end{itemize}

\vspace{1em}
\noindent\textbf{3. Logic Expressions}
\begin{itemize}
    \item $g = \overline{P1} + \overline{P2}$ (2 NOTs + 1 OR)
    \item $e = b + c$ (1 OR)
    \item $d = c + e$ (1 OR)
\end{itemize}

\vspace{1em}
\noindent\textbf{4. Pin Mapping}
\begin{itemize}
    \item \textbf{P1} – D2 (Input)
    \item \textbf{P2} – D3 (Input)
    \item \textbf{b} – D4 (Input)
    \item \textbf{c} – D5 (Input)
    \item \textbf{g LED} – D8 (Output)
    \item \textbf{e LED} – D9 (Output)
    \item \textbf{d LED} – D10 (Output)
\end{itemize}

\vspace{1em}
\noindent\textbf{5. Circuit Diagram} \\
\includegraphics[width=0.8\linewidth]{z.jpeg} % Replace with your circuit image

\end{multicols}

% Switch to single column for the rest
\newpage


\vspace{1em}
\section*{6. GitHub Code Link}
\url{https://github.com/amuru052004/Likhitha_fwc/tree/main/Hardware}

\vspace{1em}
\section*{7. Conclusion}
This document provides the successful hardware implementation of GATE Q60 using Arduino and minimal gates. The outputs g, e, and d have been verified using the logic: 2 NOT gates and 3 OR gates, and the results matched expected behavior.

\end{document}
