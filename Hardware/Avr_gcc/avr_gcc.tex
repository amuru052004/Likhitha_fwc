\documentclass[12pt]{article}
\usepackage{graphicx}
\usepackage{enumitem}
\usepackage{geometry}
\usepackage{amsmath}
\usepackage{xcolor}
\usepackage{float}
\usepackage{amsmath, amssymb}
\usepackage{enumitem}
\usepackage{float}
\usepackage{caption}
\usepackage{subcaption}
\usepackage{xcolor}
\usepackage{fancyhdr}
\usepackage{datetime2}
\usepackage{pgfplots}
\definecolor{darkskyblue}{rgb}{0.0, 0.5, 1.0}
\definecolor{skyblue}{RGB}{135, 206, 235}
\usepackage{wrapfig}
\usepackage{circuitikz}
\geometry{margin=1in}

\geometry{a4paper, margin=1in}

\begin{document}
\pagestyle{empty} % Start with empty page style

\thispagestyle{fancy} % Apply fancy style only to the first page
\fancyhf{} % Clear header and footer
\renewcommand{\headrulewidth}{0pt} % Remove header rule

\fancyhead[L]{% Left header
	\includegraphics[width=8cm, height=1.7cm]{i.png} % Adjust dimensions
}
\fancyhead[R]{% Right header
    Name: Amuru Likhitha\\
    Batch: COMETFWC019 \\
    Date: 03 june 2025 
}
\vspace{10cm}
\begin{center}
   
    {\LARGE \textbf{\textcolor{darkskyblue}{\\  GATE QUESTION \\ ECE 2009 Q39}}}
\end{center}

{\color{violet}\textbf{Question}}

Q39) What are the counting states $(Q_1, Q_2)$ for the counter shown in the figure below?

\vspace{1em}
\includegraphics[width=0.85\textwidth]{z.png}  % Replace with actual image path

\vspace{1em}
{\color{violet}\textbf{Options:}}
\begin{enumerate}[label=(\Alph*)]
    \item 11, 10, 00, 11, 10,...
    \item 01, 10, 11, 00, 01,...
    \item 00, 11, 01, 10, 00,...
    \item 01, 10, 00, 01, 10,...
\end{enumerate}

\vspace{1em}
{\color{violet}\subsection*{Answer and Explanation}}
\textbf{Answer:} (D) 01, 10, 00, 01, 10,...\\

\textbf{Explanation:}

The circuit consists of two JK flip-flops. Their inputs are connected such that:
\begin{itemize}
    \item $J_1 = Q_2$, $K_1 = 1$
    \item $J_2 = \overline{Q_1}$, $K_2 = 1$
\end{itemize}

Simulating the state transitions starting from $Q_1 = 0, Q_2 = 0$ gives the sequence:
\[
00 \rightarrow 01 \rightarrow 10 \rightarrow 00 \rightarrow ...
\]
This is a 3-state counter. The correct option that matches this sequence cyclically is option (D).

\end{document}

