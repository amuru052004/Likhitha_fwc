\documentclass[12pt]{article}
\usepackage{graphicx}
\usepackage[margin=1in]{geometry}
\usepackage{caption}
\usepackage{xcolor}
\usepackage{float}
\usepackage{amsmath}
\usepackage{array}
\usepackage{multicol}
\usepackage{hyperref}
\usepackage{url}
\usepackage{listings}
\usepackage{courier}

\lstset{
  basicstyle=\ttfamily\footnotesize,
  backgroundcolor=\color{gray!10},
  frame=single,
  breaklines=true,
  captionpos=b
}

\setlength{\columnsep}{1cm}

\begin{document}

% Title Block
\begin{figure}[H]
    \begin{minipage}{0.45\textwidth}
        \includegraphics[width=\textwidth]{i.png} % Replace with your logo
    \end{minipage} \hfill
    \begin{minipage}{0.45\textwidth}
        \textbf{Name: Amuru Likhitha} \\
        \textbf{Batch: COMETFWC019} \\
        \textbf{Date: 03 June 2025}
    \end{minipage}
\end{figure}

\begin{center}
    {\LARGE \textbf{\textcolor{cyan}{GATE 2009, ECE Question Number 39}}}
\end{center}
\vspace{1em}

\begin{multicols}{2}

\noindent\textbf{Abstract} \\[0.5em]
This project implements the GATE 2009 Q39 JK Flip-Flop counter using a Raspberry Pi Pico W, buttons, and LEDs. The system simulates a 3-state counter with the sequence \texttt{01 → 10 → 00 → 01...}. Button presses act as clock signals, and LEDs represent the outputs Q1 and Q2.

\begin{figure}[H]
    \centering
    \includegraphics[width=0.6\linewidth]{jk_flipflop.png} % Replace with JK FF logic diagram
    \caption*{Figure: JK Flip-Flop logic diagram used in simulation}
\end{figure}

\vspace{1em}
\noindent\textbf{1. Components}
\begin{table}[H]
\small
\centering
\begin{tabular}{|p{4.2cm}|c|}
\hline
\textbf{Component} & \textbf{Qty} \\
\hline
Raspberry Pi Pico W & 1 \\
Push Button (clock input) & 1 \\
LEDs (Q1, Q2) & 2 \\
220$\Omega$ Resistors & 2 \\
Breadboard & 1 \\
Jumper Wires & 8 \\
Micro-USB Cable & 1 \\
\hline
\end{tabular}
\caption*{Table 1: List of components used}
\end{table}

\vspace{1em}
\noindent\textbf{2. Setup and Connections}
\begin{itemize}
    \item \textbf{Button (Clock)} → GPIO 15 with a pull-down resistor.
    \item \textbf{LED Q1} → GPIO 16 via 220$\Omega$ resistor.
    \item \textbf{LED Q2} → GPIO 17 via 220$\Omega$ resistor.
    \item GND connections from Pico to breadboard.
    \item Button press triggers state change.
\end{itemize}

\vspace{1em}
\noindent\textbf{3. State Table for Q39}
\begin{table}[H]
\centering
\begin{tabular}{|c|c|c|}
\hline
\textbf{Press} & \textbf{Q2} & \textbf{Q1} \\
\hline
Initial & 0 & 0 \\
1st & 0 & 1 \\
2nd & 1 & 0 \\
3rd & 0 & 0 \\
4th & 0 & 1 \\
\hline
\end{tabular}
\caption*{Table 2: State transition sequence for 3-state counter}
\end{table}

\end{multicols}

\newpage


\section*{4. Analysis}
\begin{itemize}
    \item Flip-Flop 1 toggles on every clock (J=Q2, K=1)
    \item Flip-Flop 2 toggles based on Q1 (J=Q1, K=1)
    \item Implemented logic replicates JK Flip-Flop transitions
    \item LED outputs match state sequence: 01 → 10 → 00 → ...
\end{itemize}

\section*{5. Circuit Image}
\begin{figure}[H]
    \centering
    \includegraphics[width=0.7\linewidth]{avr_gcc (2).jpeg} % Replace with actual hardware image
    \caption*{Figure: Real hardware setup using Raspberry Pi Pico W}
\end{figure}

\section*{6. Conclusion}
This hardware implementation using Raspberry Pi Pico W accurately simulates the JK flip-flop logic defined in GATE ECE 2009 Q39. The LED outputs reflect the correct 3-state sequence, and button input effectively acts as a manual clock pulse generator.
\section*{7. GitHub Code Link}
\url{https://github.com/amuru052004/Likhitha_fwc/tree/main/Hardware}
\end{document}
