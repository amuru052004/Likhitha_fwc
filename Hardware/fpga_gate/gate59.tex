\documentclass[a4paper,12pt]{article}
\usepackage{times}
\usepackage{amsmath}
\usepackage{tikz}
\usepackage{amsmath,multicol}
\usepackage[a4paper,top=1cm,left=1.2in,bottom=5cm, right=1.4in,]{geometry}
\usepackage{fancyhdr}
\usepackage[dvipsnames]{xcolor}
\pagestyle{fancy}
\fancyhf{}
\renewcommand{\headrulewidth}{2.3pt} 
\renewcommand{\headrule}{{\color{MidnightBlue}\hrule width\headwidth height\headrulewidth}}
\fancyhead[L]{\includegraphics[width=5cm,height=2cm]{COMET.jpeg}} 
\fancyhead[C]{\textcolor{MidnightBlue}{\Large\textbf{GATE  \vspace{0.3em} \\ \large EC 2009-59}}}
\fancyhead[R]{\textcolor{MidnightBlue}{K SATHVIKA \\ COMET.FWC20}} 
\setlength{\headheight}{4cm}
\setlength{\headsep}{17pt}\usepackage{amsmath,multicol}
\usepackage[a4paper,top=1cm,left=1.2in,bottom=5cm, right=1.4in,]{geometry}
\usepackage{enumitem}
\usepackage{fancybox}
\usepackage{fancyhdr}
\usepackage[dvipsnames]{xcolor}
\pagestyle{fancy}
\fancyhf{}
\renewcommand{\headrulewidth}{2.3pt} 
\renewcommand{\headrule}{{\color{MidnightBlue}\hrule width\headwidth height\headrulewidth}}
\fancyhead[L]{\includegraphics[width=5cm,height=2cm]{COMET.jpeg}} 
\fancyhead[C]{\textcolor{MidnightBlue}{\Large\textbf{GATE  \vspace{0.3em} \\ \large EC 2009-39}}}
\fancyhead[R]{\textcolor{MidnightBlue}{K SATHVIKA \\ COMET.FWC20}} 
\setlength{\headheight}{4cm}
\setlength{\headsep}{17pt}

\begin{document}

Two products are sold from a vending machine, which has two push buttons $P_1$ and $P_2$.  
When a button is pressed, the price of the corresponding product is displayed in a 7-segment display.

\begin{itemize}
    \item If no buttons are pressed, ‘0’ is displayed, signifying Rs. 0.
    \item If only $P_1$ is pressed, ‘2’ is displayed, signifying Rs. 2.
    \item If only $P_2$ is pressed, ‘5’ is displayed, signifying Rs. 5.
    \item If both $P_1$ and $P_2$ are pressed, ‘E’ is displayed, signifying “Error”.
\end{itemize}

\vspace{0.5em}

The names of the segments in the 7-segment display and the glow of the display for ‘0’, ‘2’, ‘5’ and ‘E’ are shown below:

\vspace{0.5cm}


% Define each segment with rounded corners and spacing
\newcommand{\segmenta}{\draw[very thick] (0.4,2.6) -- (1.3,2.6);}
\newcommand{\segmentb}{\draw[very thick] (1.3,2.5) -- (1.3,1.6);}
\newcommand{\segmentc}{\draw[very thick] (1.3,1.4) -- (1.3,0.5);}
\newcommand{\segmentd}{\draw[very thick] (0.4,0.4) -- (1.3,0.4);}
\newcommand{\segmente}{\draw[very thick] (0.3,1.4) -- (0.3,0.4);}
\newcommand{\segmentf}{\draw[very thick] (0.3,2.6) -- (0.3,1.6);}
\newcommand{\segmentg}{\draw[very thick] (0.3,1.5) -- (1.2,1.5);}

% Define off segment (light gray)
\newcommand{\segmentaoff}{\draw[gray!30] (0.3,2.7) -- (1.1,2.6);}
\newcommand{\segmentboff}{\draw[gray!30] (1.1,2.6) -- (1.1,1.6);}
\newcommand{\segmentcoff}{\draw[gray!30] (1.1,1.4) -- (1.1,0.4);}
\newcommand{\segmentdoff}{\draw[gray!30] (0.5,0.4) -- (1.1,0.4);}
\newcommand{\segmenteoff}{\draw[gray!30] (0.3,1.4) -- (0.3,0.4);}
\newcommand{\segmentfoff}{\draw[gray!30] (0.3,2.6) -- (0.3,1.6);}
\newcommand{\segmentgoff}{\draw[gray!30] (0.3,1.5) -- (1.1,1.5);}

% Macro to draw a digit with ON/OFF segments
\newcommand{\digitseven}[8]{%
  \begin{scope}[xshift=#8 cm]
    \ifnum#1=1 \segmenta \else \segmentaoff \fi
    \ifnum#2=1 \segmentb \else \segmentboff \fi
    \ifnum#3=1 \segmentc \else \segmentcoff \fi
    \ifnum#4=1 \segmentd \else \segmentdoff \fi
    \ifnum#5=1 \segmente \else \segmenteoff \fi
    \ifnum#6=1 \segmentf \else \segmentfoff \fi
    \ifnum#7=1 \segmentg \else \segmentgoff \fi
  \end{scope}
}


\begin{center}
\begin{tikzpicture}[scale=1.3]

% Labelled segment layout
\begin{scope}[xshift=-4cm]
  \digitseven{1}{1}{1}{1}{1}{1}{1}{0}
  \node at (0.7,2.8) {a};
  \node at (1.45,2.1) {b};
  \node at (1.45,0.85) {c};
  \node at (0.7,0.2) {d};
  \node at (0.0,0.85) {e};
  \node at (0.0,2.1) {f};
  \node at (0.7,1.75) {g};
\end{scope}

% Digit 0
\node at (0.7,3) {0};
\digitseven{1}{1}{1}{1}{1}{1}{0}{0}

% Digit 2
\node at (3.0,3) {2};
\digitseven{1}{1}{0}{1}{1}{0}{1}{2.3}

% Digit 5
\node at (5.3,3) {5};
\digitseven{1}{0}{1}{1}{0}{1}{1}{4.6}

% Digit E
\node at (7.6,3) {E};
\digitseven{1}{0}{0}{1}{1}{1}{1}{6.9}

\end{tikzpicture}
\end{center}

\vspace{0.5cm}

\textbf{Consider:}
\begin{itemize}
    \item[(i)] Push button pressed / not pressed is equivalent to logic 1 / 0 respectively.
    \item[(ii)] A segment glowing / not glowing in the display is equivalent to logic 1 / 0 respectively.
\end{itemize}
\noindent
\textbf{Q.59 If segments $a$ to $g$ are considered as functions of $P_1$ and $P_2$, then which of the following is correct?}

\begin{enumerate}
    \item[(A)] $g = \overline{P_1} + P_2,\quad d = c + e$
    \item[(B)] $g = P_1 + P_2,\quad d = c + e$
    \item[(C)] $g = \overline{P_1} + P_2,\quad e = b + c$
    \item[(D)] $g = P_1 + P_2,\quad e = b + c$
\end{enumerate}

\vspace{0.5cm}
\newpage
\pagestyle{empty}
\textbf{Solution:}

\begin{itemize}
    \item $g$ segment glows for 2, 5, and E, but not for 0.
    \item That means:
    \[
    g = \overline{P_1}\cdot P_2 + P_1 \cdot \overline{P_2} + P_1 \cdot P_2 = P_1 + P_2
    \]
    \item Only option (B) and (D) satisfy $g = P_1 + P_2$.
    \item For option (B), $d = c + e$ is consistent with the ON segments for all digits (0, 2, 5, E).
\end{itemize}

\textbf{Therefore, the correct answer is: (B)}

\end{document}
