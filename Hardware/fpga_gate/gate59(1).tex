\documentclass[12pt]{article}
\usepackage{graphicx}
\usepackage[margin=1in]{geometry}
\usepackage{caption}
\usepackage{xcolor}
\usepackage{float}
\usepackage{amsmath}
\usepackage{array}
\usepackage{multicol}
\usepackage{hyperref}
\usepackage{url}
\usepackage{listings}
\usepackage{courier}

\lstset{
  basicstyle=\ttfamily\footnotesize,
  backgroundcolor=\color{gray!10},
  frame=single,
  breaklines=true,
  captionpos=b
}

\setlength{\columnsep}{1cm}

\begin{document}

% Title Block
\begin{figure}[H]
    \begin{minipage}{0.45\textwidth}
        \includegraphics[width=\textwidth]{i.png} % Replace with your image
    \end{minipage} \hfill
    \begin{minipage}{0.45\textwidth}
        \textbf{Name: Amuru Likhitha} \\
        \textbf{Batch: COMETFWC019} \\
        \textbf{Date: 10 July 2025}
    \end{minipage}
\end{figure}

\begin{center}
    {\LARGE \textbf{\textcolor{cyan}{GATE 2009, ECE Question Number 59}}}
\end{center}
\vspace{1em}

\begin{multicols}{2}

\noindent\textbf{Abstract} \\[0.5em]
This project uses an Arduino UNO and a 7-segment display to simulate a vending machine logic system from GATE 2009 Question 59. Based on two inputs (P1, P2), prices are displayed: 0, 2, 5, or E.

\vspace{1em}
\noindent\textbf{1. Components}
\begin{table}[H]
\small
\centering
\begin{tabular}{|p{4.2cm}|c|}
\hline
\textbf{Component} & \textbf{Qty} \\
\hline
Arduino UNO & 1 \\
Push Buttons (P1, P2) & 2 \\
7-Segment Display (Common Cathode) & 1 \\
220$\Omega$ Resistors & 7 \\
Breadboard & 1 \\
Jumper Wires & 12 \\
Laptop with Arduino IDE & 1 \\
\hline
\end{tabular}
\caption*{Table 1: List of components used}
\end{table}

\vspace{1em}
\noindent\textbf{2. Setup and Connections}
\begin{itemize}
    \item Connect push button P1 to D2 and P2 to D3 with pull-down resistors.
    \item Connect 7-segment display segments:
    \begin{itemize}
        \item a → D4, b → D5, c → D6, d → D7, e → D8, f → D9, g → D10
    \end{itemize}
    \item Connect each segment pin through a 220$\Omega$ resistor.
    \item Common cathode pin of display to GND.
    \item Arduino GND connected to breadboard GND.
\end{itemize}

\vspace{1em}
\noindent\textbf{3. Logic Summary}
\begin{itemize}
    \item Inputs: P1, P2 (1 = pressed, 0 = not pressed)
    \item Segment g: $g = P1 + P2$
    \item Segment e: $e = b + c$
    \item Segment d: $d = c + e$
    \item Display outputs: 0, 2, 5, or E depending on input
\end{itemize}

\vspace{1em}
\noindent\textbf{4. Pin Mapping}
\begin{itemize}
    \item \textbf{P1} – D2 (Input)
    \item \textbf{P2} – D3 (Input)
    \item \textbf{7-Segment Display:}
    \begin{itemize}
        \item a – D4, b – D5, c – D6, d – D7
        \item e – D8, f – D9, g – D10
    \end{itemize}
\end{itemize}

\vspace{1em}
\noindent\textbf{5. Circuit Diagram} \\
\includegraphics[width=0.85\linewidth]{z.png} % Replace with your circuit image

\end{multicols}

\newpage

\vspace{1em}
\section*{6. Analysis}
\subsection*{6.1 Truth Table}
\begin{table}[H]
\centering
\begin{tabular}{|c|c|c|}
\hline
\textbf{P1} & \textbf{P2} & \textbf{Display} \\
\hline
0 & 0 & 0 \\
1 & 0 & 2 \\
0 & 1 & 5 \\
1 & 1 & E \\
\hline
\end{tabular}
\end{table}

\subsection*{6.2 Segment Activation}
\begin{table}[H]
\centering
\small
\begin{tabular}{|c|c|c|c|c|c|c|c|}
\hline
\textbf{Digit} & a & b & c & d & e & f & g \\
\hline
0 & 1 & 1 & 1 & 1 & 1 & 1 & 0 \\
2 & 1 & 1 & 0 & 1 & 1 & 0 & 1 \\
5 & 1 & 0 & 1 & 1 & 0 & 1 & 1 \\
E & 1 & 0 & 0 & 1 & 1 & 1 & 1 \\
\hline
\end{tabular}
\end{table}

\subsection*{6.3 Derivations}
\begin{itemize}
    \item $g = P1 + P2$
    \item $e = b + c$
    \item $d = c + e$
\end{itemize}


\vspace{1em}
\noindent\textbf{8. GitHub Code Link} \\
\url{https://github.com/amuru052004/Likhitha_fwc/tree/main/Hardware/platformio}

\vspace{1em}
\section*{7. Conclusion}
This hardware implementation using a 7-segment display successfully models vending machine logic using basic input and output pins of Arduino UNO. The results are validated against GATE 2009 Question 59 output specifications.

\end{document}
