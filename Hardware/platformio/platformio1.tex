\documentclass[12pt]{article}
\usepackage{graphicx}
\usepackage[margin=1in]{geometry}
\usepackage{caption}
\usepackage{xcolor}
\usepackage{float}
\usepackage{amsmath}
\usepackage{array}
\usepackage{multicol}
\usepackage{hyperref}
\usepackage{url}
\usepackage{listings}
\usepackage{courier}

\lstset{
  basicstyle=\ttfamily\footnotesize,
  backgroundcolor=\color{gray!10},
  frame=single,
  breaklines=true,
  captionpos=b
}

\setlength{\columnsep}{1cm} % spacing between columns

\begin{document}

% Title Block
\begin{figure}[H]
    \begin{minipage}{0.45\textwidth}
        \includegraphics[width=\textwidth]{i.png} % <-- Your hardware image
    \end{minipage} \hfill
    \begin{minipage}{0.45\textwidth}
        \textbf{Name: Amuru Likhitha} \\
        \textbf{Batch: COMETFWC019} \\
        \textbf{Date: 6th July 2025}
    \end{minipage}
\end{figure}

\begin{center}
    {\LARGE \textbf{\textcolor{cyan}{GATE 2009, ECE Question Number 37}}}
\end{center}
\vspace{1em}

% Begin columns
\begin{multicols}{2}

% Abstract
\noindent\textbf{Abstract} \\[0.5em]
\textit{(GATE 2009, Question No. 37 – Minimum number of 2-to-1 MUXes to implement AND and XOR gates)} \\[0.5em]
This project demonstrates the logic implementation of a 2-input AND gate and a 2-input XOR gate using MicroPython and Raspberry Pi Pico.

\vspace{1em}
\noindent\textbf{1. Components}
\begin{table}[H]
\small
\centering
\begin{tabular}{|p{4.2cm}|c|}
\hline
\textbf{Component} & \textbf{Qty} \\
\hline
Raspberry Pi Pico & 1 \\
USB Cable (Micro-USB) & 1 \\
Push Buttons & 2 \\
LEDs & 2 \\
220$\Omega$ Resistors & 4 \\
Jumper Wires (M-M) & 10 \\
Breadboard & 1 \\
Laptop with Thonny IDE & 1 \\
\hline
\end{tabular}
\caption*{Table 1: List of components used}
\end{table}

\vspace{1em}
\noindent\textbf{2. Setup and Connections}
\begin{enumerate}
    \item Connect push buttons to GPIO pins GP14 and GP15.
    \item Attach LEDs to GP16 (AND output) and GP17 (XOR output) via 220$\Omega$ resistors.
    \item Use pull-down configuration for button inputs.
    \item Connect all grounds to Pico GND.
    \item Power and program the Pico using Thonny IDE.
\end{enumerate}

\vspace{1em}
\noindent\textbf{3. Implementation Steps}
\begin{enumerate}
    \item Assemble the circuit on the breadboard.
    \item Connect Pico to PC and open Thonny.
    \item Write and upload MicroPython code.
    \item Press buttons and observe LED behavior.
\end{enumerate}

\vspace{1em}
\noindent\textbf{4. Truth Table}
\[
\begin{array}{|c|c|c|c|}
\hline
A & B & A \cdot B & A \oplus B \\
\hline
0 & 0 & 0 & 0 \\
0 & 1 & 0 & 1 \\
1 & 0 & 0 & 1 \\
1 & 1 & 1 & 0 \\
\hline
\end{array}
\]

\vspace{1em}
\noindent\textbf{5. Boolean Expressions}
\begin{itemize}
    \item AND Gate: \( F_{AND} = A \cdot B \)
    \item XOR Gate: \( F_{XOR} = A \cdot \overline{B} + \overline{A} \cdot B \)
\end{itemize}

\end{multicols}

\vspace{3em}
\noindent\textbf{6. Input and Output Pins}
\begin{itemize}
    \item \textbf{A (Input)} – GP14
    \item \textbf{B (Input)} – GP15
    \item \textbf{AND Output LED} – GP16
    \item \textbf{XOR Output LED} – GP17
\end{itemize}

\vspace{1em}
\noindent\textbf{7. Circuit Screenshot} \\
\\
\includegraphics[width=0.5\textwidth]{platformio (1).jpeg} % <-- Replace with your real circuit image filename

\vspace{1em}
\noindent\textbf{8. Hardware Code – MicroPython (Thonny IDE)}
\begin{lstlisting}[language=Python]
from machine import Pin
import utime

# Define input pins
A = Pin(14, Pin.IN, Pin.PULL_DOWN)
B = Pin(15, Pin.IN, Pin.PULL_DOWN)

# Define output pins
and_led = Pin(16, Pin.OUT)
xor_led = Pin(17, Pin.OUT)

while True:
    a_val = A.value()
    b_val = B.value()

    # AND gate
    and_output = a_val and b_val
    and_led.value(and_output)

    # XOR gate
    xor_output = (a_val and not b_val) or (not a_val and b_val)
    xor_led.value(xor_output)

    print("A =", a_val, "B =", b_val, 
          "AND =", and_output, "XOR =", xor_output)

    utime.sleep(0.1)
\end{lstlisting}

\vspace{1em}
\noindent\textbf{9. GitHub Code Link} \\
\url{https://github.com/amuru052004/Likhitha_fwc/tree/main/Hardware/platformio}

\end{document}
