\documentclass[12pt]{article}
\usepackage{graphicx}
\usepackage[margin=1in]{geometry}
\usepackage{caption}
\usepackage{xcolor}
\usepackage{float}
\usepackage{amsmath}
\usepackage{array}
\usepackage{multicol}
\usepackage{hyperref}
\usepackage{url}
\usepackage{listings}
\usepackage{courier}

\lstset{
  basicstyle=\ttfamily\footnotesize,
  backgroundcolor=\color{gray!10},
  frame=single,
  breaklines=true,
  captionpos=b
}

\setlength{\columnsep}{1cm}

\begin{document}

% Title Block
\begin{figure}[H]
    \begin{minipage}{0.45\textwidth}
        \includegraphics[width=\textwidth]{i.png} % <-- Replace with your image filename
    \end{minipage} \hfill
    \begin{minipage}{0.45\textwidth}
        \textbf{Name: Amuru Likhitha} \\
        \textbf{Batch: COMETFWC019} \\
        \textbf{Date: 06 July 2025}
    \end{minipage}
\end{figure}

\begin{center}
    {\LARGE \textbf{\textcolor{cyan}{GATE 2009, ECE Question Number 38}}}
\end{center}
\vspace{1em}

\begin{multicols}{2}

\noindent\textbf{Abstract} \\[0.5em]
\textit{Simulation of latch behavior using Raspberry Pi Pico to demonstrate NAND and NOR latch transitions for the input combinations (0,1) → (1,1).}

\vspace{1em}
\noindent\textbf{1. Components}
\begin{table}[H]
\small
\centering
\begin{tabular}{|p{4.2cm}|c|}
\hline
\textbf{Component} & \textbf{Qty} \\
\hline
 Pico2w & 1 \\
Push Buttons & 2 \\
LEDs & 2 \\
220$\Omega$ Resistors & 4 \\
Breadboard & 1 \\
Jumper Wires & 10 \\
Laptop with Thonny IDE & 1 \\
\hline
\end{tabular}
\caption*{Table: Components used}
\end{table}

\vspace{1em}
\noindent\textbf{2. Setup}
\begin{itemize}
    \item GP15: Input P1 (Push Button)
    \item GP14: Input P2 (Push Button)
    \item GP16: NAND Q Output (LED)
    \item GP17: NOR Q Output (LED)
    \item GND and VBUS properly connected
\end{itemize}

\vspace{1em}
\noindent\textbf{3. Observation}
\begin{itemize}
    \item \textbf{NAND Latch:} (0,1) → (1,0) → holds at (1,0)
    \item \textbf{NOR Latch:} (0,1) → (1,0) → transitions to (0,0)
\end{itemize}

\end{multicols}
\\
\noindent\textbf{4. Truth Tables}\\
\\
\\
\vspace{2em}
\begin{minipage}{0.45\linewidth}
\centering
\textbf{NAND Latch} \\
\[
\begin{array}{|c|c|c|}
\hline
P1 & P2 & \text{Output (Q1, Q2)} \\
\hline
0 & 1 & (1, 0) \\
1 & 1 & (1, 0) \text{ (holds)} \\
\hline
\end{array}
\]
\end{minipage}
\hfill
\begin{minipage}{0.45\linewidth}
\centering
\textbf{NOR Latch} \\
\[
\begin{array}{|c|c|c|}
\hline
P1 & P2 & \text{Output (Q1, Q2)} \\
\hline
0 & 1 & (1, 0) \\
1 & 1 & (0, 0) \\
\hline
\end{array}
\]
\end{minipage}

\vspace{14em}
\noindent\textbf{5. Circuit Image} \\
\\
\includegraphics[width=0.6\linewidth]{platformio (1).jpeg}

\vspace{1em}
\noindent\textbf{6. Hardware Code – MicroPython}
\begin{lstlisting}[language=Python]
from machine import Pin
import utime

P1 = Pin(15, Pin.IN, Pin.PULL_DOWN)
P2 = Pin(14, Pin.IN, Pin.PULL_DOWN)

nand_q = Pin(16, Pin.OUT)
nor_q = Pin(17, Pin.OUT)

nand_q_state = 1
nor_q_state = 1

while True:
    p1 = P1.value()
    p2 = P2.value()

    if p1 == 0 and p2 == 1:
        nand_q_state = 1
    elif p1 == 1 and p2 == 1:
        nand_q_state = nand_q_state

    if p1 == 0 and p2 == 1:
        nor_q_state = 1
    elif p1 == 1 and p2 == 1:
        nor_q_state = 0

    nand_q.value(nand_q_state)
    nor_q.value(nor_q_state)

    print("P1 =", p1, "P2 =", p2, 
          "NAND Q =", nand_q_state, 
          "NOR Q =", nor_q_state)

    utime.sleep(0.2)
\end{lstlisting}

\vspace{1em}
\noindent\textbf{7. GitHub Code Link} \\
\url{https://github.com/amuru052004/Likhitha_fwc/tree/main/Hardware/assembly}

\vspace{1em}
\noindent\textbf{8. Conclusion}

This project successfully demonstrates latch behavior for NAND and NOR gates using MicroPython and Raspberry Pi Pico. 

\end{document}
