\documentclass[12pt]{article}
\usepackage{graphicx}
\usepackage{amsmath}
\usepackage{geometry}
\usepackage{xcolor}
\usepackage{float}
\usepackage{amsmath, amssymb}
\usepackage{float}
\usepackage{caption}
\usepackage{subcaption}
\usepackage{xcolor}
\usepackage{fancyhdr}
\usepackage{datetime2}
\usepackage{pgfplots}
\definecolor{darkskyblue}{rgb}{0.0, 0.5, 1.0}
\definecolor{skyblue}{RGB}{135, 206, 235}
\usepackage{wrapfig}
\usepackage{circuitikz}
\geometry{margin=1in}

\begin{document}
\pagestyle{empty} % Start with empty page style

\thispagestyle{fancy} % Apply fancy style only to the first page
\fancyhf{} % Clear header and footer
\renewcommand{\headrulewidth}{0pt} % Remove header rule

\fancyhead[L]{% Left header
	\includegraphics[width=8cm, height=1.7cm]{i.png} % Adjust dimensions
}
\fancyhead[R]{% Right header
    Name: Amuru Likhitha\\
    Batch: COMETFWC019 \\
    Date: 03 june 2025 
}
\vspace{10cm}
\begin{center}
   
    {\LARGE \textbf{\textcolor{darkskyblue}{\\  GATE QUESTION \\ ECE 2009 Q37}}}
\end{center}
\vspace{-1cm} %adjust vertical space
\vspace{1em}
\noindent
{\color{violet}\textbf{Question}}

\vspace{0.5em}
\noindent
Refer to the NAND and NOR latches shown in the figure below.

\begin{center}
    \includegraphics[width=0.8\textwidth]{gate5.png} % Latch diagram
\end{center}

\vspace{0.5em}
\noindent
The inputs \((P_1, P_2)\) for both the latches are first made \((0,1)\), and then, after a few seconds, made \((1,1)\). The corresponding stable outputs \((Q_1, Q_2)\) are:

\vspace{0.5em}
\noindent
\textbf{Options:}
\begin{itemize}
    \item[(A)] NAND: first \((0,1)\) then \((0,1)\); \quad NOR: first \((1,0)\) then \((0,0)\)
    \item[(B)] NAND: first \((1,0)\) then \((1,0)\); \quad NOR: first \((1,0)\) then \((1,0)\)
    \item[(C)] NAND: first \((1,0)\) then \((1,0)\); \quad NOR: first \((1,0)\) then \((0,0)\)
    \item[(D)] NAND: first \((1,0)\) then \((1,1)\); \quad NOR: first \((0,1)\) then \((0,1)\)
\end{itemize}

\vspace{1em}
\noindent
{\color{violet}\textbf{Answer and Explanation}}

\vspace{0.5em}
{\color{!50!black}\textbf{Answer: (C)}}

\vspace{0.5em}
\textbf{NAND Latch:}
\begin{itemize}
    \item With inputs \((0,1)\): the latch stabilizes to \((Q_1, Q_2) = (1,0)\)
    \item With inputs \((1,1)\): it remains at \((1,0)\)
\end{itemize}

\textbf{NOR Latch:}
\begin{itemize}
    \item With inputs \((0,1)\): the latch stabilizes to \((Q_1, Q_2) = (1,0)\)
    \item With inputs \((1,1)\): it transitions to \((0,0)\)
\end{itemize}

\textbf{Final outputs:}
\begin{itemize}
    \item NAND Latch: \((1,0) \rightarrow (1,0)\)
    \item NOR Latch: \((1,0) \rightarrow (0,0)\)
\end{itemize}

\end{document}


